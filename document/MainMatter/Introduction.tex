\chapter*{Introducción}\label{chapter:introduction}
\addcontentsline{toc}{chapter}{Introducción}
En la actualidad, debido a los avances en la tecnología, la información suele almacenarse de manera digital. Los modelos de Machine Learning(ML) surgen para 
ayudar a los humanos a procesar e interpretar esta información, la cual luego es utilizada para resolver problemas de la vida real. Estos modelos 
son empleados en ciencias como la biología y la medicina, en la identificación de células cancerígenas, en el análisis de genoma y creación de nuevos fármacos. 
Además, en otros sectores como la educación, construcción y robótica [\cite{39}]. También son empleados en la detección de fraude [\cite{4}], en el procesamiento de 
imágenes [\cite{1},\cite{3}], para solucionar problemas sociales y económicos [\cite{2}]. Sin los modelos de ML sería casi imposible realizar estas tareas sin 
gastar demasiados recursos de software y tiempo.

Si bien estos algoritmos de aprendizaje han automatizado el proceso de toma de decisiones, aún tienen mucha dependencia hacia los humanos. 
Los especialistas deben escoger de un conjunto de modelos, aquel con características más similares a la definición de su problema. Los datos de entrada 
deben ser procesados según el tipo de estructura que soporta el modelo. Además, cada uno puede tener tanto parámetros continuos como discretos y se debe seleccionar 
una configuración de los mismos que permita resolver el problema con buenos resultados.

Una de las limitaciones de su utilización es que cada vez es más difícil suplir la demanda de expertos que realicen este trabajo. El entrenamiento de los modelos de 
ML requiere de un gran dominio de temas como probabilidades, estadísticas y programación. Además de que es necesario invertir mucho tiempo en el proceso.

Los sistemas de aprendizaje automático(AutoML) surgen para resolver estos inconvenientes. Estos construyen canalizaciones de ML de forma automática describiendo un 
flujo de tareas a realizar. Estas tareas son: el procesamiento de datos, selección de modelo y optimización de hiperparámetros entre otras. Todas esas etapas se 
realizan sin intervención de los humanos, al menos eso es lo que describe la definición de AutoML. La realidad es que aún necesitan un poco de ayuda externa, sobre 
todo en el proceso de limpieza de los datos [\cite{36}].

Los softwares AutoML varían en características como el de espacio de búsqueda, el cual incluye todas las posibles soluciones del problema que pretenden solucionar. 
La estrategia de optimización de dicho espacio que permite buscar soluciones eficientes. Por último, la estrategia para estimar el rendimiento de cada una de las 
canalizaciones que forman, la cual posibilita comparar dos soluciones y escoger la mejor[\cite{33},\cite{37},\cite{52}].

Estos sistemas se emplean como baseline, ya que la solución obtenida quizás no es la óptima, pero es capaz de hacer predicciones cercanas al óptimo de un 
problema en cuestión. Un experto con el empleo de los mismos podría identificar qué tipo de modelos, parámetros, serían buenos para atacar su problema inicial. 
Una persona sin muchos conocimientos de ML se beneficia de su simplicidad, lo que le permite una fácil y rápida utilización.

AutoML acota la intervención de los humanos en la creación de canalizaciones de ML, pero en su mayoría son muy lentos. Existen tareas irresolubles para los 
sistemas producto a limitantes relacionadas con el dominio de aplicación de sus datos de entrada. Este dominio puede que necesite técnicas de procesamiento que el 
sistema de aprendizaje desconoce o que sus algoritmos obtienen malos resultados en ese ámbito de aplicación.

La problemática anterior conduce a un nuevo enfoque de AutoML: AutoML heterogéneo. Este permite encontrar la mejor forma de transformar una entrada en una 
salida deseada. Con ellos, tanto la entrada como la salida pueden tener variada estructura y abarcar varios tipos de datos, ya sean imágenes, textos, datos 
tabulares y la mezcla de los mismos [\cite{33}].

\begin{flushleft} 
{\Large { \textbf{Motivación} }}
\end{flushleft}
Los marcos de AutoML heterogéneo disminuirían aún más el trabajo de los usuarios que lo utilicen. Además, aumentaría el número de 
tareas que tendrían solución sin importar la estructura o el área al que pertenezcan los datos de entrada de los sistemas. Con todas las ventajas 
que resaltan a la vista con el nuevo enfoque surge la necesidad de formas de evaluación para ver si los resultados que aportan son correctos.


\begin{flushleft} 
    {\Large { \textbf{Antecedentes}}}
\end{flushleft}
Desde el surgimiento de los modelos de machine learning se han hecho notar las investigaciones sobre la construcción de conjuntos de datos de evaluación. 
Existen muchos documentos científicos que proponen softwares para medir rendimiento en tareas de ML [\cite{1},\cite{3},\cite{2},\cite{7}]. A estos sistemas se les 
denomina Benchmark.

Los investigadores de AutoML apoyados en los Benchmark de ML han aportado puntos de referencia [\cite{10},\cite{15},\cite{31}] y conjuntos de datos [\cite{28}]. 
El objetivo de cada uno de estos sistemas es evaluar y comparar eficiencia en igualdad de condiciones. Además, se nutren de sitios que posibilitan la obtención de los 
medios necesarios para cumplir con sus objetivos: OpenML[\cite{43}], Kaggle[\cite{44}] y UCI[\cite{45}].

\begin{flushleft} 
    {\Large { \textbf{Problemática}}}
\end{flushleft}
Los benchmark de AutoML poseen muchas limitaciones en sus conjuntos de datos. Estos tienen poca similitud con los problemas de la vida real. Además, suelen 
pertenecer a un solo dominio y solo explotan un mínimo de todas las funcionalidades de los sistemas AutoML. En caso de que incluyan más de un dominio de aplicación, 
se desprenden de su significado semántico. También muchos de estos conjuntos se crearon para sistemas antecedentes y puede que sean demasiado fáciles para los AutoML 
de nueva generación.

Las comparaciones de los softwares de aprendizaje automático que realizan los benchmark, muchas veces sufren de falta de equidad. Además, tienen fallas en su 
metodología producto de una incorrecta investigación de la naturaleza de los dataset que utilizan. Más allá de los conjuntos que usualmente son empleados en las 
evaluaciones, existen otros, pero por carecer de uniformidad en su estructura y por necesitar procesamiento son apartados.


\textbf{Hipótesis}: Asumiendo que es posible que los sistemas AutoML sean heterogéneos y que se podrá establecer una comparación lo más 
justa posible entre los mismos.

\begin{flushleft} 
    {\Large {\textbf{Objetivos}}}
\end{flushleft}

\textbf{Objetivo General}: Construir un benchmark que cuantifique la eficiencia de los sistemas AutoML, que sus conjuntos pertenezcan a distintos dominios y tareas 
de ML. Los datos deben ser un desafío debido a sus meta-características.


\begin{flushleft} 
\textbf{Objetivos Específicos:}
\begin{itemize}
    \item Realizar un estudio que permita identificar las características, aportes y fallas de algunos benchmark y comparaciones de AutoML que se recogen en la 
    literatura. 
    Los benchamrk de machine learning que aporten características importantes también serán incluidos como objeto de estudio.
    \item Recopilación de conjuntos de datos que evalúen la heterogeneidad de los sistemas.
    \item Implementación de un sistema que permita descargar dichos datos y los transforme a un formato común.  
    \item Evaluación de AutoMLs seleccionados.
    \item Estudio de los resultados obtenidos.
\end{itemize}
\end{flushleft} 
