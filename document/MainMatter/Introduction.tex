\chapter*{Introducción}\label{chapter:introduction}
\addcontentsline{toc}{chapter}{Introducción}
La inteligencia artificial juega un papel importante en el proceso de digitalización y automatización de la sociedad. El aprendizaje de máquinas (ML) es una rama de 
investigación de la inteligencia artificial que ayuda a los humanos a resolver problemas de sectores de la industria y de ciencias como la biología y la medicina 
[\cite{39}]. Los modelos que esta rama agrupa, son utilizados en tareas como la detección de fraude [\cite{4}], el procesamiento de imágenes [\cite{1},\cite{3}], 
para solucionar problemas sociales y económicos [\cite{2}] entre muchas otras. Cada uno de los algoritmos de ML son un ascenso en la automatización del proceso de 
toma de decisiones y en el análisis inteligente de la información. 

A pesar de la gran popularidad de estos algoritmos, tienen mucha dependencia hacia los humanos. Los expertos en aprendizaje de máquinas son los encargados 
de seleccionar el mejor, en dependencia de las características del problema que desea resolver. Además, es también tarea de los especialistas, encontrar 
una mejor solución, mediante la optimización de los parámetros de entrada que poseen cada uno de estos algoritmos. 

Con la creciente popularidad del aprendizaje de máquinas, es difícil suplir la demanda de expertos que tengan un amplio conocimiento del tema [\cite{33}]. 
Las tareas que involucra el entrenamiento de un modelo de ML, necesitan de muchos recursos de tiempo por parte de los humanos y de software para su realización.

Como parte de la solución a estos inconvenientes surgen nuevas técnicas que permiten reducir el costo de recursos humanos y de tiempo. Entre estas,
los sistemas de aprendizaje de máquinas automáticos(Auto-ML), los cuales realizan una construcción de flujos de soluciones automáticas basadas en técnicas de ML, 
para la resolución de problemas de la sociedad. Los sistemas Auto-ML varían en características como las propiedades de su espacio de búsqueda de soluciones y la 
estrategia de optimización que sigue para recorrer este espacio de manera eficiente. El espacio de búsqueda está formado por todos los algoritmos que 
implementa el sistema. Otra característica es la forma en que estima el rendimiento de cada una de las soluciones que encuentra [\cite{33},\cite{37},\cite{52}], dicha 
característica le permite quedarse con la mejor solución posible.

Estos sistemas pretenden emplearse como baseline, ya que realizan predicciones bastante cercanas a la solución de las tareas que resuelven. Un experto, con el empleo de 
los mismos, podría identificar qué tipo de modelos y parámetros, serían buenos para atacar su problema inicial. Las personas sin muchos conocimientos de ML, pueden 
emplearlos, pues son simples y fáciles de manipular.

Las herramientas Auto-ML disminuyen la intervención de los humanos en comparación con los modelos de ML [\cite{36}]. Sin embargo; estas herramientas son muy lentas, y 
existen tareas para la cuales no encuentran solución. La obtención de soluciones en estos sistemas, se encuentra limitada principalmente por el dominio de aplicación de 
los datos del problema y por la estructura que presentan. Estas limitaciones están relacionadas con la forma en que los sistemas reciben los datos de entrada, las técnicas 
de procesamiento que incluyen y los algoritmos que implementan. 

Las deficiencias anteriores conducen a crear nuevos enfoques que permitan desplegar una herramienta Auto-ML en diferentes dominios de aplicación, en distintas tareas 
con variada estructura y variados tipos de entrada. Este despliegue debe realizarse sin la intervención de los humanos en el procesamiento de los datos de entrada y 
sin grandes modificaciones en la estructura que los datos presentan. Para ello, surge una nueva línea de investigación de las herramientas Auto-ML que permite encontrar 
la mejor forma de transformar una entrada en una salida deseada, con el objetivo de diversificar la estructura y los tipos que se reciben como entrada, ya sean imágenes, 
textos, datos tabulares y la mezcla de los mismos. A este nuevo enfoque se le denomina Auto-ML heterogéneo [\cite{33}].

\begin{flushleft} 
{\Large { \textbf{Motivación} }}
\end{flushleft}
Los marcos de Auto-ML heterogéneo disminuirían el esfuerzo de los usuarios que lo utilicen. Además, muchos problemas tendrían solución sin grandes modificaciones en 
la estructura de sus datos y sin pérdida de las propiedades de su dominio de aplicación. Con todas las ventajas que resaltan a la vista con el nuevo enfoque, 
surge la necesidad de formas de evaluación para ver si los resultados que aportan son correctos.

\begin{flushleft} 
    {\Large { \textbf{Antecedentes}}}
\end{flushleft}
Desde el surgimiento de los modelos de aprendizaje de máquina se han hecho notar las investigaciones sobre la construcción de conjuntos de datos para evaluación. 
Existen muchos documentos científicos que proponen softwares para medir rendimiento en tareas de ML [\cite{1},\cite{3},\cite{2},\cite{7}], son denominados benchmark.

Los investigadores de Auto-ML apoyados en los benchmark de ML han aportado puntos de referencia [\cite{10},\cite{15},\cite{31}] y conjuntos de datos [\cite{28}] para 
evaluar y comparar la eficiencia de los sistemas en igualdad de condiciones.  Estos se nutren de sitios que posibilitan la obtención de sus conjuntos de datos de 
evaluación: OpenML[\cite{43}], Kaggle[\cite{44}] y UCI[\cite{45}].

\begin{flushleft} 
    {\Large { \textbf{Problemática}}}
\end{flushleft}
Los benchmark de Auto-ML poseen muchas limitaciones en sus conjuntos de datos. Estos son transformados a una estructura común con el fin de poder ser evaluados en los 
sistemas. Estas transformaciones provocan pérdida en las propiedades de los conjuntos; tales como el tipo de sus datos originales y el significado que poseen para el 
dominio al que pertenecen. También, muchos de los conjuntos utilizados, se crean para evaluar las funcionalidades de sistemas antecesores y puede que no cuenten con 
todo el rigor necesario para evaluar los Auto-ML de nueva generación.

Las evaluaciones de los sistemas Auto-ML presenten en los benchmark, muchas veces sufren de errores en su metodología que provocan una desigual comparación del 
desempeño de las herramientas. Los errores más comunes, surgen por una incorrecta investigación de las propiedades de sus conjuntos y por disparidad 
en las configuraciones de las evaluaciones. Más allá de los conjuntos que usualmente son empleados en las evaluaciones, existen otros, pero por carecer de uniformidad 
en su estructura y por necesitar procesamiento, son apartados.


\textbf{Hipótesis}: Asumiendo que es posible que los sistemas Auto-ML sean heterogéneos y que se podrá establecer una comparación lo más 
justa posible entre los mismos.

\begin{flushleft} 
    {\Large {\textbf{Objetivos}}}
\end{flushleft}

\textbf{Objetivo General}: Construir un benchmark que cuantifique la eficiencia de los sistemas Auto-ML en escenarios heterogéneos, incluyendo conjuntos que tengan 
variedad en sus metacaracterísticas, tipos de datos, dominio al que pertenece y tareas que resuelven. 

\begin{flushleft} 
\textbf{Objetivos Específicos:}
\begin{itemize}
    \item Realizar un estudio de los benchmark en la literatura.
    \item Recopilar conjuntos de datos que evalúen la heterogeneidad de los sistemas.
    \item Implementar un sistema que permita descargar dichos datos y los retorne a un formato común.  
    \item Evaluar sistemas Auto-ML del estado del arte.
    \item Estudiar los resultados que se obtienen en las evaluaciones.
\end{itemize}
\end{flushleft} 
