\begin{annexes}\label{chapter:annexes}
   
En este apartado se ofrece una descripción de cada uno de los conjuntos de datos que pertenecen al benchmark \textit{HAutoML-Bench}.

\begin{flushleft} 
    { \textbf{Conjunto: SST-en}}\label{description:sst-en}
\end{flushleft}

Sentimiento Sentiment Treebank [\cite{sst}]: Es un conjunto de datos que contiene 215 154 frases con etiquetas de opinión detalladas (5 clases). 
La tarea es predecir de cada instancia relacionada a la opinión sobre una película el sentimiento que se manifiesta desde más negativa a más positiva 
(clasificación multiclase).

\begin{flushleft} 
    { \textbf{Conjunto: Vaccine-es}}\label{description:vaccine}
\end{flushleft}

La tarea VaxxStance [\cite{VaxxStance}] forma parte de la competencia IberLEF 2021[\cite{iberlef2021}]. El objetivo principal es detectar la postura en las redes 
sociales sobre un tema muy controvertido y de moda, como es el movimiento Antivacunas. La tarea final es determinar si un tweet expresa una postura a favor, 
en contra o neutral (ninguno) hacia un tema previamente definido. La selección incluye solo el conjunto del idioma español. 


\begin{flushleft} 
    { \textbf{Conjunto: Wikiann-es}}\label{description:wikkian}
\end{flushleft}

WikiANN (a veces llamado PAN-X) [\cite{wikkian}] es un conjunto de datos de reconocimiento de entidades nombradas multilingüe que consta de artículos de Wikipedia 
expresados en secuencias de tokens y anotados con etiquetas LOC (ubicación), PER (persona) y ORG (organización) en el formato IOB2. Cada una de las etiquetas se 
encuentran transformadas a números. 
La versión original admite 176 idiomas, la selección abarca solo el idioma español.  

\begin{flushleft} 
    { \textbf{Conjuntos: Paws-x-es y Paws-x-en}}\label{description:paws}
\end{flushleft}

Paws-x [\cite{paws-x}] el conjunto original posee instancias de dos oraciones traducidos del inglés por humanos y por máquinas en seis idiomas tipológicamente distintos. La tarea es 
identificación de paráfrasis\footnote{Frase que expresa el mismo contenido que otra pero con diferente estructura sintáctica.} a partir de una clasificación binaria. 
La selección incluye solo los conjuntos del idioma español y el inglés. 

\begin{flushleft} 
    { \textbf{Conjunto: Language-Identification}}\label{description:languaje}
\end{flushleft}

El conjunto de datos de identificación de idioma [\cite{languaje-identification}] es una colección de 90 000 muestras que consisten en la entrada de texto y la 
etiqueta de idioma correspondiente. 
Este conjunto de datos se crea mediente la recopilación de datos de otras 3 fuentes. La tarea es entrenar un modelo para la identificación de idiomas, que es una tarea 
de clasificación de texto de varias clases.

\begin{flushleft} 
    { \textbf{Conjunto: Wikicat-es}}\label{description:wikicat}
\end{flushleft}

WikiCAT-es [\cite{wikicat}] es un corpus español para tareas de clasificación de textos temáticos. Se crea automáticamente a partir de fuentes de Wikipedia y Wikidata 
clasificados en 19 categorías diferentes. 

\begin{flushleft} 
    { \textbf{Conjunto: Price-Book}}\label{description:price}
\end{flushleft}

El conjunto [\cite{price-book}] es una gran base de datos de libros. Libros de diferentes géneros, de miles de autores. La tarea es utilizar el conjunto de datos para predecir el precio de 
los libros en función de un conjunto determinado de características. Entre estas se encuentran: el título, autor, edición, las opiniones y calificaciones de los 
clientes, la sinopsis, el departamento donde suele estar categoría del libro y el género. 


\begin{flushleft} 
    { \textbf{Conjunto: Predict-Salary}}\label{description:predict}
\end{flushleft}

El conjunto [\cite{predict-salary}] trata sobre los trabajos de los científico de datos en la India en 2017. La tarea consiste en predecir el rango del salario según características del 
trabajo como años de experiencia que necesita, descripción, designación de tareas, tipo, locación, entre otras. Se traduce a una clasificación multiclase donde el rango 
del salario es la etiqueta a predecir.

\begin{flushleft} 
    { \textbf{Conjunto: Inferes}}\label{description:inferes}
\end{flushleft}

InferES[\cite{inferes}], un corpus original para la inferencia del lenguaje natural (NLI) en español europeo. Contiene 8.055 pares de texto. En la tarea es determinar el 
el significado de la relación que se da entre dos textos. La selección posee tres vías; dado una hipótesis y una premisa se determina si existe una vinculación, una contradicción o
una relación neutral entre ellas.

\begin{flushleft} 
    { \textbf{Conjunto: Wnli-es}}\label{description:wnli}
\end{flushleft}


Este conjunto de datos[\cite{wnli2}] es una traducción profesional al español del conjunto de datos Winograd NLI[\cite{wnli}]. Los mismos presentan un esquema de Winograd, en donde su 
par de oraciones de entrada difieren en solo una o dos palabras. Además que contienen una ambigüedad que se resuelve de manera opuesta en las dos oraciones y requiere 
el uso del conocimiento del mundo y el razonamiento para su resolución.

\begin{flushleft} 
    { \textbf{Conjuntos: Wikineural-es y Wikineural-en}}\label{description:wikineural}
\end{flushleft}

El original WikiNEuRal [\cite{wikineural},\cite{wikineural2}] consiste en una técnica novedosa que se basa en una base de conocimiento léxico multilingüe (es decir, BabelNet) y arquitecturas basadas en 
transformadores para producir anotaciones de alta calidad para NER (Named Entity Recognition) multilingüe. Se generaron automáticamente datos de entrenamiento para 
NER en 9 idiomas. La selección incluye solo los conjuntos del idioma español e inglés.

\begin{flushleft} 
    { \textbf{Conjunto: Meddocan}}\label{description:meddocan}
\end{flushleft}

Meddocan[\cite{meddocan}] es un problema de reconocimiento de entidades parte de la competencia IberLEF 2019 [\cite{sepln-2019iberlef}]. Tiene el objetivo de 
detectar información sensible en documentos médicos de habla española. 
EL conjunto de datos se compone de 1 000 casos clínicos de estudio donde 750 se utilizan para entrenamiento y 250 para pruebas. La tarea es reconocer la entidad y 
etiquetarla con el formato IOB2.

\begin{flushleft} 
    { \textbf{Conjunto: HAHA}}\label{description:haha}
\end{flushleft}

La campaña de evaluación HAHA[\cite{haha}] forma parte de la competición IberLEF 2019 [\cite{sepln-2019iberlef}] propone diferentes subtareas relacionadas con la 
detección automática de humor. A partir del corpus comentado de tweets en español proporcionado se pretende clasificar cada tweet en una broma o no (humor 
intencionado por el autor).

\begin{flushleft} 
    { \textbf{Conjunto: Fraudulent-Jobs}}\label{description:fraudulent}
\end{flushleft}

Este conjunto de datos [\cite{fraudulent-job}] contiene 18 000 descripciones de puestos de trabajo, de las cuales unas 800 son falsas. Los datos consisten tanto en información textual como en 
metainformación sobre los trabajos. La tarea es identificar si los mismos son fraudulentos o no. Contiene entradas de texto, numéricas categóricas, relacionadas a las 
caracteristicas que tiene un trabajo como el salario,el departamento, la descripción, etc. 


\begin{flushleft} 
    { \textbf{Conjunto: Stroke-Predictions}}\label{description:stroke}
\end{flushleft}

Según la Organización Mundial de la Salud (OMS), el accidente cerebrovascular es la segunda causa de muerte en todo el mundo, responsable de aproximadamente el 
11\% del total de muertes.
Este conjunto de datos [\cite{stroke-prediction}] se utiliza para predecir si es probable que un paciente sufra un accidente cerebrovascular en función de los 
parámetros de entrada como el sexo, la edad, diversas enfermedades y el tabaquismo. Cada fila de los datos proporciona información relevante sobre el paciente.

\begin{flushleft} 
    { \textbf{Conjunto: Project-Kickstarter}}\label{description:project}
\end{flushleft}

Predecir si un proyecto de Kickstarter [\cite{project-kickstarter}] propuesto logrará el objetivo de financiación en función de las características de cada uno. Posee características de texto, 
como el título, la descripción de cada proyecto. Las características numéricas, como la cantidad de dinero solicitada, la fecha de publicación. Las características 
categóricas, como el país, la moneda, etc. Este conjunto de datos representa una tarea compleja en la que los modelos deben considerar las interacciones entre las 
modalidades para abordar una cuestión central del negocio de Kickstarter.

\begin{flushleft} 
    { \textbf{Conjunto: Spanish-Wine}}\label{description:wine}
\end{flushleft}

Este conjunto de datos [\cite{spanish-wine}] está relacionado con las variantes rojas de los vinos españoles. El conjunto de datos describe varias métricas de popularidad y descripción y 
su efecto en su calidad . Los conjuntos de datos se pueden utilizar para tareas de clasificación o regresión. La tarea escogida (regresión) es predecir los precios 
utilizando los datos proporcionados.

\begin{flushleft} 
    { \textbf{Conjunto: Human-Bot}}\label{description:human}
\end{flushleft}

Este conjunto de datos [\cite{human-bot},\cite{Human-Bot}] está compuesto por 37438 filas correspondientes a diferentes cuentas de usuarios en Twitter. Cada fila contiene 20 columnas que son las funciones 
recopiladas a través de la API de Twitter. Estas varían entre categorías, textos , url y datos de tiempo, relacionadas a las características de las páginas de los 
usuarios. La tarea es predecir si el usuario es un bots o un humano. 


\begin{flushleft} 
    { \textbf{Conjunto: Women-Clothing}}\label{description:women}
\end{flushleft}

El conjunto [\cite{women-clothing}] trata sobre datos del comercio electrónico de ropa de mujer que gira entorno a las reseñas escritas por los clientes. La tarea es predecir el nombre 
(categoría) a la que corresponde un producto, para ello se apoyan en nueve característica más. Entre estas se encuentran las calificaciones, el título de la reseña , 
el nombre del departamento del producto, el texto de la reseña, la edad de los revisores, si recomiendan o no la prenda entre otras. 

\begin{flushleft} 
    { \textbf{Conjunto: Sentiments-Lexicons}}\label{description:sentiments}
\end{flushleft}

El conjunto [\cite{sentiments-lexicons}] está formado por palabras con sentimientos positivos (buenos, excelente, increíble) y negativos (malos, asquerosos, terribles) 
para 81 idiomas. La tarea es etiquetar cada una de estas en positiva y negativa para idioma español. 

\begin{flushleft} 
    { \textbf{Conjuntos: Stsb-es y Stsb-en}}\label{description:stsb}
\end{flushleft}

Stsb-Multi[\cite{stsb},\cite{stsb2}] recoge las traducciones multilingües y el original en inglés del conjunto de datos STSbenchmark. 
Este conjunto de datos proporciona pares de oraciones y una puntuación de su similitud entre las mismas un float ente 0.0 y 5.0. La selección incluye solo los conjuntos 
del idioma español e inglés. 

\begin{flushleft} 
    { \textbf{Conjunto: Google-Guest}}\label{description:google}
\end{flushleft}

Los datos de la competencia Google QUEST Q\&A Labeling [\cite{google-quest}] incluyen preguntas y respuestas de varias propiedades de StackExchange. Su tarea es predecir 
el valor de una etiqueta de 30 que se pueden utilizar para cada par de preguntas y respuestas. Cada fila contiene una sola pregunta y una sola respuesta a esa pregunta, junto con 
funciones adicionales. Los datos de entrenamiento contienen filas con algunas preguntas duplicadas (pero con respuestas diferentes). Los datos de la prueba no contienen 
preguntas duplicadas. Este no es un desafío de predicción binaria. Las etiquetas objetivo se agregan a partir de múltiples evaluadores y pueden tener valores continuos 
en el rango [0,1]. Por lo tanto, las predicciones también deben estar en ese rango. 

\begin{flushleft} 
    { \textbf{Conjunto: PUBHEALTH}}\label{description:pubhealth}
\end{flushleft}

PUBHEALTH [\cite{pubhealth}] es un conjunto de datos integral para la verificación de hechos automatizada explicable de afirmaciones de salud pública. Cada instancia en el conjunto de 
datos de PUBHEALTH tiene una etiqueta de veracidad asociada (verdadero, falso, no probado, mixto). Además, cada instancia en el conjunto de datos tiene un campo de 
texto de explicación. La explicación es una justificación por la cual a la afirmación se le ha asignado una etiqueta de veracidad particular. Además tiene otras 
informaciones como la fecha de creación , el sitio de donde proviene la información y el autor.

\begin{flushleft} 
    { \textbf{Conjunto: Trec}}\label{description:trec}
\end{flushleft}

El conjunto de datos [\cite{trec3}] de clasificación de preguntas de la Conferencia de recuperación de texto (TREC)[\cite{trec2},\cite{trec1}] tiene como tarea 
clasificar la idea esencial de la pregunta en una categoría de 6 existentes.

\begin{table}[H]
    \centering
    \resizebox{15cm}{!} {
    \begin{tabular}{|c|c|c|c|c|c|}
    \hline 
    Conjuntos de                & Tipo de    & Dominio & Datos de  & Datos de  & Cantidad de \\
      Datos                     &  Tarea  &            & Entrenamiento & Prueba & Columnas\\
    \hline
    paws-x-en                 & Binaria & Texto       &  49401 & 4000 & 3 \\       
    paws-x-es                 & Binaria & Texto       &  49401 & 4000 & 3   \\          
    wnli-es                   & Binaria & Texto       &  635   & 70   & 3   \\     
    stroke-prediction         & Binaria & multimodal  &  4088  & 1022    & 11 \\
    fraudulent-jobs           & Binaria & multimodal  &  12516 & 5364    & 17 \\         
    project-kickstarter       & Binaria & multimodal  & 108129 & 63465   & 12 \\
    haha                      & Binaria & Texto       & 24000  & 6000    & 2 \\
    sentiment-lexicons-es     & Binaria & Texto       &  4075   & 200    & 2 \\
    twitter-human-bots        & Binaria & multimodal  &  29950  & 7488   & 18 \\
    wikicat-es                & Muliclase & Texto      &  7909   & 3402  & 2 \\
    sst-en                    & Muliclase & Texto      &  8544   & 2210  & 2  \\             
    women-clothing            & Muliclase & multimodal &  16440  & 7046  & 9 \\   
    inferes                   & Muliclase & multimodal &  6444  & 1612  & 6 \\
    predict-salary            & Muliclase & multimodal &  19802  & 6601  & 8 \\
    language-identification   & Muliclase & Texto      &  80000  & 10000 & 2 \\
    vaccine-es                & Muliclase & Texto      &  2003   & 694  & 2 \\
    pub-health                & Muliclase & multimodal &  9831   & 1235 & 8 \\
    trec                      & Muliclase & Texto      &  5452   & 500  & 2 \\
    spanish-wine              & regresión & multimodal &  6000   & 1500 & 11 \\
    stsb-en                   & regresión & Texto      &  7249   & 1379 & 3  \\
    stsb-es                   & regresión & Texto      &  7249   & 1379 & 3 \\
    price-book                & regresión & multimodal &  6237   & 1560 & 9 \\
    google-guest              & regresión & multimodal &  6079   & 476  & 40 \\
    wikiann-es                & NER & Texto  &  30000   & 10000 & 2 \\
    meddocan                  & NER & Texto  &  25622   & 8432 & 2 \\
    wikineural-es             & NER & Texto  &  76320   & 19158 & 2 \\
    wikineural-en             & NER & Texto  &  92720   & 23187 & 2 \\
    \hline
    \end{tabular}
    \caption{Propiedades de los conjuntos de datos: tipo de tarea, dominio, cantidad de instancias de entrenamiento y de prueba, cantidad de columnas}
    \label{tab:data}
    }
\end{table}
  
\end{annexes}

