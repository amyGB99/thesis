\begin{resumen}
La inteligencia artificial continua en avance y con ella los sistemas de aprendizaje automático (AutoML). Estos sistemas extienden sus 
funcionalidades con técnicas novedosas con el fin de resolver una mayor cantidad de problemas de la vida real con un buen rendimiento.
Debido a esto se hace necesario medir el desempeño de cada uno de los sistemas de nueva generación. Este trabajo presenta \textit{HAutoML-Bench}, un
benchmark que cuantifica el rendimiento de las herramientas de aprendizaje automático en escenarios heterogéneos. En su construcción se estudian sus antecesores
para recopilar estrategias y evitar errores. Se presentan todas las estrategias seguidas para su formación y se analiza la efectividad de cada una de ellas. 
Además, se efectúan experimentaciones que incluyen evaluaciones cualitativas y cuantitativas de sistemas AutoML presentes en el estado del arte.

\end{resumen}

\begin{abstract}
Artificial intelligence continues to advance and with it machine learning systems (AutoML). These systems extend their
functionalities with novel techniques in order to solve a greater number of real-life problems with good performance.
Due to this, it is necessary to measure the performance of each of the new generation systems. This paper presents \textit{HAutoML-Bench}, a benchmark that quantifies 
the performance of machine learning tools in heterogeneous scenarios. In its construction, its predecessors are studied to compile strategies and avoid errors. All the 
strategies followed for their training are presented and the effectiveness of each of them is analyzed.
In addition, experiments are carried out that include qualitative and quantitative evaluations of AutoML systems present in the state of the art.
\end{abstract}