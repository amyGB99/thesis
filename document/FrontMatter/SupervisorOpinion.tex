\begin{opinion}
    En la actualidad el Aprendizaje Automático ha llegado a todas las ramas de la industria, ayudando a resolver un gran número de problemas pero 
creando la necesidad de un enorme número de expertos para poder utilizar las herramientas adecuadas en cada caso.
En este escenario el AutoML propone una solución ayudando con la selección de forma automática de las mejores soluciones con el problema añadido de que incrementa
el costo computacional ya que tiene que evaluar muchas soluciones para resolver cada problema. Realizando esta tarea cada vez.
El área de investigación en que incursiona la estudiante propone un enfoque para que los sistemas de AutoML puedan evaluarse teniendo en cuenta la heterogeneidad de los problemas a resolver.

La estudiante Amanda González en esta investigación se adentra en un tema del estado del arte de gran actualidad y para eso tuvo que utilizar conocimientos de varias asignaturas de la carrera y otros que no son parte del currículum estándar.
Su propuesta implicó un fuerte estudio del estado del arte para conocer tanto los sistemas de AutoML como los benchmarks utilizados para evaluarlos, así como Implementar un sistema que permita descargar dichos datos y los convierta a un formato común.  Además tuvo que evaluar sistemas AutoML del estado del arte y estudiar los resultados obtenidos. 
Como resultado final obtiene su propio benchmark donde evaluar sistemas de AutoML según las características más importantes seleccionadas por el propio estudiante.

Para poder afrontar el trabajo, la estudiante tuvo que revisar literatura científica relacionada con la temática así como soluciones existentes y bibliotecas de software que pueden ser apropiadas para su utilización. Todo ello con sentido crítico, determinando las mejores aproximaciones y también las dificultades que presentan.
Todo el trabajo fue realizado por la estudiante con una elevada constancia, capacidad de trabajo y habilidades, tanto de gestión, como de desarrollo y de investigación.
Por estas razones pedimos que le sea otorgada a la estudiante Amanda González Borrell la máxima calificación y, de esta manera, pueda obtener el título de Licenciado en Ciencia de la Computación.
\newpage
\centering
Tutores: 

Dr. Suilan Estévez Velarde

Lic. Ernesto Luis Estevanell Valladares 

\begingroup
\centering
\wildcard{Dr. Suilan Estévez Velarde}
\hspace{1.5cm}
\wildcard{Lic. Ernesto L. Estevanell Valladares}
\par
\endgroup
\end{opinion}